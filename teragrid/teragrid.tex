%\documentclass{beamer}
\documentclass[handout]{beamer}

\usepackage{pgfpages} 

%\setbeameroption{show only notes}

\usetheme{default}

\mode<presentation> {
%  \usetheme{Warsaw}
  \usetheme{Frankfurt}
%  \usetheme{Boadilla}
%  \usetheme{Marburg}
}

\mode<handout>{\setbeamercolor{background canvas}{bg=black!5} %
    \pgfpagesuselayout{4 on 1}[letterpaper,border shrink=4mm,landscape] %
    \setbeameroption{show notes}}

\title[CAC Intro] {TeraGrid}
\author{Brock Palen\\ \texttt{brockp@umich.edu}}
\date{TBD}

\begin{document}
  \setbeamercovered{transparent}  
  \begin{frame}
    \titlepage
    \url{http://cac.engin.umich.edu/teragrid/}
  \end{frame}

%table of contents
  \begin{frame}{Outline}
    \tableofcontents
  \end{frame}
  
  \section{About}
  \subsection {About}
  \begin{frame}{TeraGrid}
   \begin{block}{About}
    TeraGrid is an open scientific discovery infrastructure combining leadership class resources at eleven partner sites to create an integrated, persistent computational resource.
   \end{block}
   \begin{block}{Short Form}
    \begin{itemize}
     \item<2-> Lots of Compute
     \item<3-> Lots of Storage
     \item<4-> Powerful Support
    \end{itemize}
   \end{block}
  \end{frame}

  \begin{frame}{Support}
   \begin{block}{Local Support}
    \begin{itemize}
     \item The CAC is the campus-wise local resource for TerGrid support, and U of M user advocate on TG
     \item \url{tg-support@umich.edu}
     \item \url{http://cac.engin.umich.edu/teragrid/}
    \end{itemize}
   \end{block}
   \begin{block}{TG Wide}
    \begin{itemize}
     \item \url{help@teragrid.org}
     \item \url{http://teragrid.org/userinfo/}
    \end{itemize}
   \end{block}
  \end{frame}

  \section{Resources}
  
  \subsection{Compute}
  \begin{frame}{Resources}
   \begin{block}{Compute}
    \begin{itemize}
      \item Over 16 Compute (Batch) systems available
      \item Range from 0.34 TFlops to 608 TFlops
      \item Special resources such as GPU's and Cell CPUs available
      \item Shared memory, Distributed memory clusters, and Grid systems
    \end{itemize}
   \end{block}
  \end{frame}

  \subsection{Storage and Vis}
  \begin{frame}{Resources}
   \begin{block}{Storage}
     \begin{itemize}
      \item Each site/system has local scratch and NFS/Lustre home directories.
      \item Each site/system has own quota/purge policy
      \item Archive storage (Longterm/huge storage) is available at several sites, most are over 1PetaByte in size
     \end{itemize}
   \end{block}
   \begin{block}{Visulization}
   \begin{itemize}
    \item TeraGrid also gives access to severl Vis clusters and walls
    \item Not as easy to use due to location but is available
   \end{itemize}
   \end{block}
  \end{frame}

  \section{Getting Allocations}
  \begin{frame}{Allocations}
    Submit all requests via POPS: \\
    \url{https://pops-submit.teragrid.org/}
   \begin{block}{Statup/Education Allocations}
    \begin{itemize}
     \item Upto 200,000 Service Units (SUs)
     \item Up to 5TB Disk and 25TB Tape/Archive
     \item Granted all year long, for quick turn around, and startups
    \end{itemize}
   \end{block}
   \begin{block}{Research Allocations}
    \begin{itemize}
     \item Allocations larger than Startups
     \item Granded at intervals during the year
     \item Requires formal proposal
    \end{itemize}
     \url{http://teragrid.org/userinfo/access/allocations.php}
   \end{block}
  \end{frame}
\end{document}
